\documentclass{article}

\usepackage{fancyhdr}
\usepackage{extramarks}
\usepackage{amsmath}
\usepackage{amsthm}
\usepackage{amsfonts}
\usepackage{amssymb}
\usepackage{graphicx}
\usepackage{caption,subcaption}
\usepackage{subfig}
\usepackage{enumerate}          % For enumerates indexed by letters
\usepackage{bm}                 % For bold letters
\usepackage{algorithm2e}        % For pseudocode
\usepackage{url}                % So texttt wraps instead of creating hbox

%
% Basic Document Settings
%

\topmargin=-0.45in
\evensidemargin=0in
\oddsidemargin=0in
\textwidth=7in
\textheight=9.0in
\headsep=0.25in
\linespread{1.1}
\pagestyle{fancy}

\lhead{\hmwkAuthorName}
\chead{\hmwkClass:\ \hmwkTitle}
\rhead{\firstxmark}
\lfoot{\lastxmark}
\cfoot{\thepage}

\renewcommand\headrulewidth{0.4pt}
\renewcommand\footrulewidth{0.4pt}

\setlength\parindent{0pt}

%
% Create Problem Sections
%

\newcommand{\enterProblemHeader}[1]{
    \nobreak\extramarks{}{Problem \arabic{#1} continued on next page\ldots}\nobreak{}
    \nobreak\extramarks{Problem \arabic{#1} (continued)}{Problem \arabic{#1} continued on next page\ldots}\nobreak{}
}

\newcommand{\exitProblemHeader}[1]{
    \nobreak\extramarks{Problem \arabic{#1} (continued)}{Problem \arabic{#1} continued on next page\ldots}\nobreak{}
    \stepcounter{#1}
    \nobreak\extramarks{Problem \arabic{#1}}{}\nobreak{}
}

\setcounter{secnumdepth}{0}
\newcounter{partCounter}
\newcounter{homeworkProblemCounter}
\setcounter{homeworkProblemCounter}{1}
\nobreak\extramarks{Problem \arabic{homeworkProblemCounter}}{}\nobreak{}

%
% Homework Problem Environment
%
% This environment takes an optional argument. When given, it will adjust the
% problem counter. This is useful for when the problems given for your
% assignment aren't sequential. See the last 3 problems of this template for an
% example.
%
\newenvironment{homeworkProblem}[1][-1]{
    \ifnum#1>0
        \setcounter{homeworkProblemCounter}{#1}
    \fi
    \section{Problem \arabic{homeworkProblemCounter}}
    \setcounter{partCounter}{1}
    \enterProblemHeader{homeworkProblemCounter}
}{
    \exitProblemHeader{homeworkProblemCounter}
}

%
% Homework Details
%   - Title
%   - Due date
%   - Class
%   - Section/Time
%   - Instructor
%   - Author
%

\newcommand{\hmwkTitle}{CSE 547 Homework 2}
\newcommand{\hmwkDueDate}{April 26, 2017}
\newcommand{\hmwkClass}{CSE 547}
\newcommand{\hmwkAuthorName}{Brian de Silva}

%
% Title Page
%

\title{
    \vspace{2in}
    \textmd{\textbf{\hmwkClass:\ \hmwkTitle}}\\
    \normalsize\vspace{0.1in}\small{Due\ on\ \hmwkDueDate\ }\\
    \vspace{3in}
}

\author{\textbf{\hmwkAuthorName}}
\date{}

\renewcommand{\part}[1]{\textbf{\large Part \Alph{partCounter}}\stepcounter{partCounter}\\}

%
% Various Helper Commands
%

% Useful for algorithms
\newcommand{\alg}[1]{\textsc{\bfseries \footnotesize #1}}

% For derivatives
\newcommand{\deriv}[1]{\frac{\mathrm{d}}{\mathrm{d}x} (#1)}

% For partial derivatives
\newcommand{\pd}[2]{\frac{\partial}{\partial #1} (#2)}

\newcommand{\pdd}[2]{\frac{\partial #1}{\partial #2}}

% Integral dx
\newcommand{\dx}{\mathrm{d}x}

% Alias for the Solution section header
\newcommand{\solution}{\textbf{\vskip 0.2cm \large Solution:\\}}

% Useful commands
\newcommand{\bbm}{\begin{bmatrix}}
\newcommand{\ebm}{\end{bmatrix}}
\newcommand{\R}{\mathbb{R}}
\newcommand{\dtdx}{\frac{\Delta t}{\Delta x}}
\newcommand{\half}{\frac12}
\newcommand{\norm}[1]{\left\|#1\right\|}
% \newcommand{\Pr}{\mathbb{P}}


\begin{document}

\maketitle

\pagebreak

\section*{Collaborators}
I collaborated with Weston Barger and Emily Dinan on ***

% Problem 1
\begin{homeworkProblem}
	{\bf Gaussian Random Projections and Inner Products [10 Points]}
	\\
	\\
	In this problem, you will show that inner products are approximately preserved using random projections. Let $\phi(x) = \tfrac{1}{\sqrt{m}}Ax$ represent our random projection of $x \in \mathbb{R}^d$, with $A$ an $m \times d$ projection matrix with each entry sampled i.i.d from $N(0,1)$. (Note that each row of $A$ is a random projection vector, $v^{(i)}$.)  

	The \emph{norm preservation theorem} states that for all $x \in \mathbb{R}^d$, the norm of the random projection $\phi(x)$ approximately maintains the norm of the original $x$ with high probability:
	\begin{equation}
	\label{eq:npt}
	\Pr\left((1 - \epsilon) \norm{x}^2 \leq \norm{\phi(x)}^2 \leq (1 +
	\epsilon)\norm{x}^2 \right) \geq 1 - 2e^{-(\epsilon^2 -\epsilon^3)m/4},
	\end{equation}
	where $\epsilon \in (0, 1/2)$. 

	Using the norm preservation theorem, prove that for any $u,v \in \mathbb{R}^d$ s.t. $\norm{u} \leq 1$ and $\norm{v} \leq 1$,
	\begin{equation}
	\Pr(|u\cdot v - \phi(u)\cdot\phi(v)| \geq \epsilon) \leq 4e^{-(\epsilon^2 -\epsilon^3)m/4}.
	\end{equation}
	Note that $u\cdot v$ is the original dot product, and $\phi(u)\cdot \phi(v)$ is the dot product for the random projections. This statement puts a probabilistic bound on the distance between the two dot products. (\emph{Hint: Think about using Theorem (\ref{eq:npt}) with $x = u+v$ and $x = u-v$}).

	\solution

	First notice that
	\begin{align*}
		&(1-\epsilon)\|x\|^2\leq \norm{\phi(x)}^2\leq (1+\epsilon)\norm{x}^2\\
		\iff& \norm{x}^2-\epsilon\norm{x}\leq \norm{\phi(x)}^2\leq \norm{x}^2+\epsilon\norm{x}^2\\
		\iff & -\epsilon\norm{x}^2\leq \norm{\phi(x)}^2-\norm{x}^2\leq \epsilon\norm{x}^2\\
		\iff & \left|\norm{\phi(x)}^2-\norm{x}^2 \right|\leq \epsilon\norm{x}^2.
	\end{align*}
	Multiplying the above by $-1$ gives a similar statement:
	\[
		-\epsilon\norm{x}^2\leq \norm{x}^2- \norm{\phi(x)}^2\leq \epsilon\norm{x}^2.
	\]
	Observe also that $\phi$ is a linear function, so for any $x,y\in\R^d$, $\phi(x+y)=\phi(x)+\phi(y)$. We will need this fact later.
	Following the hint and applying Theorem (\ref{eq:npt}) to $x=u+v$ and $x=u-v$, we obtain
	\[
		\Pr\left((1-\epsilon)\|u+v\|^2\leq \norm{\phi(u+v)}^2\leq (1+\epsilon)\norm{u+v}^2\right) \geq 1-2e^{-(\epsilon^2-\epsilon^3)m/4}
	\]
	and
	\[
		\Pr\left((1-\epsilon)\|u-v\|^2\leq \norm{\phi(u-v)}^2\leq (1+\epsilon)\norm{u-v}^2\right) \geq 1-2e^{-(\epsilon^2-\epsilon^3)m/4}.
	\]
	Manipulating the inequalities as above, these two statements become
	\[
		\Pr\left(\left|\norm{\phi(u+v)}^2-\norm{u+v}^2 \right|\leq \epsilon\norm{u+v}^2 \right)\geq 1-2e^{-(\epsilon^2-\epsilon^3)m/4}
	\]
	and
	\[
		\Pr\left(\left|\norm{\phi(u-v)}^2-\norm{u-v}^2 \right|\leq \epsilon\norm{u-v}^2 \right)\geq 1-2e^{-(\epsilon^2-\epsilon^3)m/4}.
	\]
	These expressions are equivalent to the following
	\[
		\Pr\left(\left|\norm{\phi(u+v)}^2-\norm{u+v}^2 \right|\geq \epsilon\norm{u+v}^2 \right)\leq 1 - \left(1-2e^{-(\epsilon^2-\epsilon^3)m/4}\right) = 2e^{-(\epsilon^2-\epsilon^3)m/4}
	\]
	and
	\[
		\Pr\left(\left|\norm{\phi(u-v)}^2-\norm{u-v}^2 \right|\geq \epsilon\norm{u-v}^2 \right)\leq 1 - \left(1-2e^{-(\epsilon^2-\epsilon^3)m/4}\right) = 2e^{-(\epsilon^2-\epsilon^3)m/4}.
	\]

	Now, recall that for any two events $A$ and $B$, $\Pr(A\wedge B)\leq \Pr(A)+\Pr(B)$. Letting $A$ be the event that $$A:~~\left|\norm{\phi(u+v)}^2-\norm{u+v}^2 \right|\geq \epsilon\norm{u+v}^2$$ and $B$ be the event $$B:~~\left|\norm{\phi(u-v)}^2-\norm{u-v}^2 \right|\geq \epsilon\norm{u-v}^2,$$ it follows that the probability of both occuring is at most $\Pr(A)+\Pr(B)\leq4e^{-(\epsilon^2-\epsilon^3)m/4}$. We saw previously that 
	\begin{align*}
		&\left|\norm{\phi(u+v)}^2-\norm{u+v}^2 \right|\geq \epsilon\norm{u+v}^2\\
		\iff & -\epsilon\norm{u+v}^2\leq \norm{u+v}^2-\norm{\phi(u)+\phi(v)}^2\leq \epsilon\norm{u+v}^2
	\end{align*}
	and 
	\begin{align*}
		&\left|\norm{\phi(u-v)}^2-\norm{u-v}^2 \right|\geq \epsilon\norm{u-v}^2\\
		\iff & -\epsilon\norm{u-v}^2\leq \norm{\phi(u)-\phi(v)}^2 - \norm{u-v}^2\leq \epsilon\norm{u+v}^2.
	\end{align*}
	If both inequalities hold simultaneously, we may add them together to obtain
	\[
		-\epsilon\left(\norm{u+v}^2+\norm{u-v}^2\right) \leq \norm{\phi(u)-\phi(v)}^2-\norm{\phi(u)+\phi(v)}^2 + \norm{u+v}^2-\norm{u-v}^2\leq\epsilon\left(\norm{u+v}^2+\norm{u-v}^2\right).
	\]
	Using the Parallelogram law and Polarization identity, we can simplify this to
	\begin{align*}
		&-2\epsilon\left(\norm{u}^2+\norm{v}^2\right)\leq 4\left(u\cdot v-\phi(u)\cdot\phi(v) \right) \leq 2\epsilon\left(\norm{u}^2+\norm{v}^2\right)\\
		\iff& \epsilon\left(\norm{u}^2+\norm{v}^2\right)\leq 2\left(u\cdot v-\phi(u)\cdot\phi(v) \right) \leq \epsilon\left(\norm{u}^2+\norm{v}^2\right) \\
		\implies & 2\left|u\cdot v-\phi(u)\cdot\phi(v) \right|\leq \epsilon \left(\norm{u}^2+\norm{v}^2\right) \leq \epsilon(1+1) = 2\epsilon\\
		\implies& \left|u\cdot v-\phi(u)\cdot\phi(v) \right| \leq \epsilon.
	\end{align*}
	Putting everything together, we have our result
	\begin{align*}
		\Pr\left(\left|u\cdot v-\phi(u)\cdot\phi(v) \right| \leq \epsilon \right) \leq 4e^{-(\epsilon^2-\epsilon^3)m/4}.
	\end{align*}
\end{homeworkProblem}


\begin{homeworkProblem}
  {\bf KD-Trees and LSH for Approximate Neighbor Finding [15 Points]}
  \\
  \\In this problem, you will use KD-trees and Locality Sensitive
  Hashing (LSH) to find exact and approximate nearest neighbors. To
  explore the performance of KD-trees, you can use the starter code in
  ``NearestNeighbor.zip'' on the course website. You may alter the class
  \texttt{KDTreeAnalysis} to generate a dataset and return the time it
  takes to query from the KD-tree on that dataset, although this is not a
  part of the assignment.
  For approximate nearest neighbor search, $\alpha$ can
  be set to a value greater than 1. The KD-tree implementation was
  downloaded from\\http://sourceforge.net/projects/java-ml.

  For datasets with high dimensionality $d$, KD-trees will not scale
  well and other methods must be used. LSH offers a method for
  reducing the dimensionality of the data while still enabling
  approximate neighbor finding. For this portion of the problem, you
  will be using an artificially generated dataset consisting of
  term-document frequency for a vocabulary of size $d = 1000$ and $n =
  100000$ documents. The data can be found in ``sim\_docdata.zip'' on
  the course website. The zip file consists of two files,
  sim\_docdata.mtx and test\_docdata.mtx, which contain the term
  frequencies in a sparse matrix in Matrix Market format: each row is
  in the form ``termid docid frequency''. test\_docdata.mtx contains a
  set of documents to use for querying nearest neighbors. The size of
  the test set is 500 documents. You will need to complete the
  classes \texttt{MethodComparison},
  \texttt{GaussianRandomProjection}, and
  \texttt{LocalitySensitiveHash}.
  \begin{enumerate}[(a)]
  \item Implement a nearest neighbor search using
    LSH and $m$, the number of projections, in $\{5, 10,
    20\}$. Although normally we would search 'until time runs out',
    for this problem, just search all bins that have a Hamming
    distance from the bin of the query data point $<= 3$. Record the
    average query time and the average distance to the nearest
    neighbor for the test set. Compare this with the average query
    time and average distance using a KD-tree with $\alpha$ in $\{1,
    5, 10\}$. Note that the KD-tree implementation will be slow, and
    may take 5-10 minutes. Explain the trends found.
  \item Implement a Gaussian random projection on
    the document data for $m$ in $\{5, 10, 20\}$. Use these projections
    as the entries for a KD-tree. This results in an alternative approximate
    nearest neighbor search rather than using $\alpha > 1$. Again,
    record the average query time and average distance over the test
    set, and explain the trends found.
  \end{enumerate}

  \solution

  \begin{enumerate}[(a)]
  	\item 
  \end{enumerate}
\end{homeworkProblem}


\begin{homeworkProblem}
	{\bf The Clustering Toolkit: K-Means and EM}
	\\
	\\In this problem, you will implement the k-means algorithm and EM to fit a Gaussian Mixture Model (GMM). This problem consists of 3 parts. In part 1, you will implement the two clustering algorithms and
	evaluate on a 2D synthetic dataset. In part 2, we will provide you
	with a small text dataset (from the BBC). Your task is to 
	implement (or reuse) the same algorithms to cluster the documents. In the last part, you will implement the k-means 
	algorithm in Hadoop MapReduce, and test your algorithm against a
	subset of a Wikipedia dataset.
	\\
	\\{\bf 1. 2D synthetic data}
	\\
	\\The k-means algorithm is an intuitive way to explore the structure
	of a dataset. Suppose we are given points $x^{1}$, ..., $x^{n}$ $\in R^{2}$
	and an integer $K > 1$, and our goal is to minimize the within-cluster sum of squares
	\[
	J(\mu,\mathcal{Z})=\sum_{i=1}^{n}\left\|x^{i}-\mu_{z^{i}}\right\|_2^2,
	\]
	where $\mu=(\mu_{1},...,\mu_{K})$ are the cluster centers with the same
	dimension of data points, and $\mathcal{Z}=(z^{1},...,z^{n})$ are the cluster
	assignments, $z^{i}\in\{1,...,K\}$. One common algorithm for finding an 
	approximate solution is Lloyd's algorithm. To apply the 
	Lloyd's k-means algorithm one takes a guess at the number of clusters 
	(i.e., select a value for $K$) and initializes cluster centers $\mu_1,\ldots,\mu_K$ by 
	picking $K$ points (often randomly from $x^{1},...,x^{n}$).
	In practice, we often repeat multiple runs of Lloyd's algorithm with
	different initializations, and pick the best resulting 
	clustering in terms of the k-means objective. The algorithm then proceeds
	by iterating through two steps:

	\begin{itemize}
	\item[i.] Keeping $\mu$ fixed, find the cluster classification $\mathcal{Z}$ to minimize
	$J(\mu,\mathcal{Z})$ by assigning each point to the cluster to which it is closest. 

	\item [ii.] Keeping $\mathcal{Z}$ fixed, find new centers of the clusters $\mu$ for
	the $(m+1)^{th}$ step to minimize $J(\mu,\mathcal{Z})$ by averaging points within
	a cluster from the points in a cluster at the $m^{th}$ step.
	\end{itemize}

	Terminate the iteration to settle on $K$ final clusters if applicable.

	\begin{enumerate}[(a)]
		\item Assuming that the tie-breaking rule used in step (i) is consistent,
		does Lloyd's algorithm always converge in a finite number of steps?
		Briefly explain why.

		\item Implement Lloyd's algorithm on the two dimensional data points
		in data file ``2DGaussianMixture.csv''. Run it with some of your randomly
		chosen initialization points with different values of $K \in \{2, 3,
		5, 10,15, 20\}$, and show the clustering plots by color. 

		\item Implement Lloyd\textquoteright{}s algorithm. Run it 20 times,
		each time with different initialization of $K$ cluster centers picked
		at random from the set $\{x^{1},...,x^{n}\}$, with $K = 3$ clusters,
		on the two dimensional data points in data file 2DGaussianMixture.csv (Link is the ``Synthetic Data'' in the homework section). 
		Plot in a single figure the original data (in gray), and all $20\times3$
		cluster centers (in black) given by each run of Lloyd\textquoteright{}s
		algorithm. Also, compute the minimum, mean, and standard deviation of the within-cluster
		sums of squares for the clusterings given by each of the 20 runs.

		\item  K-means++ is another initialization algorithm for K-means (by David
		Arthur and Sergei Vassilvitskii). The algorithm proceeds as follows:
		\begin{itemize}
			\item [i.] Pick the first cluster center $\mu_{1}$ uniformly at random from
			the data $x^{1},...,x^{n}$. 
			\item [ii.] For $j=2,...,K:$ 
				\begin{itemize}
				\item For each data point, compute its distance $D_{i}$ to the nearest
				cluster center picked in a previous iteration:
			\[
			D_{i}=\min_{j^{\prime}=1,...,j-1}\left\Vert x^{i}-\mu_{j^{\prime}}\right\Vert .
			\]
				\item Pick the cluster center $\mu_{j}$ at random from $x^{1},...,x^{n}$
				with probabilities proportional to $D_{1}^{2},...,D_{n}^{2}$. 
				\end{itemize}
			\item [iii.] Return $\mu$ as the initial cluster assignments for Lloyd\textquoteright{}s
			algorithm.
		\end{itemize}  
		Replicate Part (c) using k-means++ as the initialization
		algorithm, instead of picking $\mu$ uniformly at random.

		\item Based on your observations in Part (c) and Part (d), is Lloyd's algorithm
		sensitive to initialization? 

		\item One shortcoming of k-means is that one has to specify the value
		of $K$. Consider the following strategy for picking $K$ automatically:
		try all possible values of $K$ and choose $K$ that minimizes $J(\mu,\mathcal{Z})$.
		Argue why this strategy is a good/bad idea. Suggest an alternative
		strategy.

		\item The two-dimensional real data in the file ``2DGaussianMixture.csv''
		are generated from a mixture of Gaussians with three components. Implement
		EM for general mixtures of Gaussians (not just k-means).  Initialize
		the means with the same procedure as in the k-means++ case.
		Initialize the covariances with the identity matrix.  Run your
		implementation on the synthetic dataset, and: 
		\begin{enumerate}

			\item Plot the likelihood of the data over EM iterations.  

			\item After convergence, plot the data, with the most likely
			   cluster assignment of each data point indicated by its color.  Mark
			   the mean of each Gaussian and draw the covariance ellipse for each Gaussian.  

			\item How do these results compare to those obtained by k-means?  Why?

		\end{enumerate}

	\end{enumerate}

	\solution 



	{\bf 2. Clustering the BBC News  [20 Points]}
	\\
	The dataset we will consider comes from the BBC (http://mlg.ucd.ie/datasets/bbc.html).
	The preprocessed dataset consists of the term-document frequency of 99 vocabulary and 1791 documents chosen from 5 categories: business, entertainment, politics, sport and tech.

	From lecture, we learned that the term frequency vector is not a good metric because of its biases to frequent terms. 
	Your first task is to convert the term frequency into tfidf using the following equations:

	\begin{eqnarray}
	  \text{tf}(t, d) &=&   \frac{f(t,d)}{max \{f(w,d): (w \in d)\}} \\
	  \text{idf}(t, D) &=&  \log \frac{|D|}{|\{d \in D : t \in d\}|} \\
	  \text{tfidf}(t,d,D) &=& tf (t,d) \times idf(t,D)
	\end{eqnarray}
	\begin{enumerate}[(a)]
	  \item Convert the term-doc-frequency matrix into a term-doc-tfidf matrix.
	    For each term $t$, take the average tfidf over each class $C_i
	    =\{\mbox{documents in class $i$}\}$: 
	    \[
	    \text{avg\_tfidf} (t, C_i, D) = \frac{1}{|C_i|}\sum_{d \in C_i} \text{tfidf}(t,d,D)
	    \] 
	    \textbf{For each class $C_i$, report the 5 terms with the highest $AvgTfidf$ for the class (e.g Tech: spywar:0.69, aol:0.58, $\dots$ Business: $\dots$).} 

	  \item Run k-means with $K = 5$ for 5 iterations, using the centers in ``*.centers'' for initialization.\\
	    \textbf{Plot the classification error (0/1 loss) versus the number
	      of iterations.}  \emph{Note: Don't use the optimal mapping
	      procedure from the previous question; you should keep the
	      class ids as they were in the initialization file.}

	  \item Run EM with $K = 5$ for 5 iterations. Using ``*.centers'' as the mean and identity as the covariance of the initial clusters. Initialize $\pi_{1,\dots,K}$ uniformly.\\
	    You need to be careful when updating the covariance matrix $\Sigma_k$ during the M-step. In particular, the MLE can be ill-conditioned because the data is sparse. To handle this case,
	    we can perform a shrinkage on the MLE: $\hat{\Sigma} =  (1-\lambda)\hat{\Sigma}_{MLE} + \lambda I$, which is equivalent to a MAP estimate of the posterior distribution with some prior.
	    In this problem, please use $\lambda = 0.2$. \\
	    \textbf{Plot the classification error (0/1 loss) versus number of iterations.} \\
	    \textbf{Plot the log-likelihood versus the number of iterations.}
	\end{enumerate}

	\solution

	{\bf 3. Scaling up K-Means with MapReduce  [35 Points]}
	\\

	The k-means algorithm fits naturally in the MapReduce Framework. Specifically, each iteration of k-means corresponds to one MapReduce cycle: 
	\begin{itemize}
	  \item The map function maps each document with key being the cluster id.
	  \item During the shuffling phase, each documents with the same cluster id will be sent to the same reducer.
	  \item The reduce function reduces on the clusterid and updates the cluster center.
	\end{itemize}

	Because we do not have the ground truth labels for this dataset, the evaluation will be done on the k-means objective function:
	\begin{eqnarray}
	  J(\mathcal{Z},\mu) &=&  \sum_{i=1}^N  \|x^i - \mu_{z^i}\|_2^2 \\
	  (\mathcal{Z}^*, \mu^*) &=& \arg\min_{\mathcal{Z}, \mu} J(\mathcal{Z},\mu)
	\end{eqnarray}
	where $z^i$ is the cluster assignment of example $i$, $\mu_j$ is the center of cluster $j$. 

	To get the value of $J(\mathcal{Z},\mu)$ for each iteration, the reducer with key $j$ will also compute and output the sum of the squares of $L_2$ distance in cluster $j$. At the end of each iteration, you can examine the mean and standard deviation of the clusters. 

	Run k-means with $K = 20$ for 5 iterations. Use the initial cluster centers from ``center0.txt''.
	\begin{enumerate}
	  \item  \textbf{Plot the objective function value $J(\mathcal{Z},\mu)$  versus the number of iterations.}
	  \item  \textbf{For each iteration, report mean (top 10 words in tfidf) of the largest 3 clusters  (by total squared $L_2$ distance). (Use the printClusters function in the starter code.)}
	  % \item  \textbf{Assuming you have 64 machines, without
	  %   actually running experiments, try to estimate the relative runtime
	  %   among $K = 10, 100, 10000$. What is the bottleneck of each
	  %   setting?}
	 
	\end{enumerate}

\end{homeworkProblem}

\end{document}