\documentclass{article}

\usepackage{fancyhdr}
\usepackage{extramarks}
\usepackage{amsmath}
\usepackage{amsthm}
\usepackage{amsfonts}
\usepackage{amssymb}
\usepackage{graphicx}
\usepackage{caption,subcaption}
\usepackage{subfig}
\usepackage{enumerate}          % For enumerates indexed by letters
\usepackage{bm}                 % For bold letters
\usepackage{algorithm2e}        % For pseudocode
\usepackage{url}                % So texttt wraps instead of creating hbox

%
% Basic Document Settings
%

\topmargin=-0.45in
\evensidemargin=0in
\oddsidemargin=0in
\textwidth=7in
\textheight=9.0in
\headsep=0.25in
\linespread{1.1}
\pagestyle{fancy}

\lhead{\hmwkAuthorName}
\chead{\hmwkClass:\ \hmwkTitle}
\rhead{\firstxmark}
\lfoot{\lastxmark}
\cfoot{\thepage}

\renewcommand\headrulewidth{0.4pt}
\renewcommand\footrulewidth{0.4pt}

\setlength\parindent{0pt}

%
% Create Problem Sections
%

\newcommand{\enterProblemHeader}[1]{
    \nobreak\extramarks{}{Problem \arabic{#1} continued on next page\ldots}\nobreak{}
    \nobreak\extramarks{Problem \arabic{#1} (continued)}{Problem \arabic{#1} continued on next page\ldots}\nobreak{}
}

\newcommand{\exitProblemHeader}[1]{
    \nobreak\extramarks{Problem \arabic{#1} (continued)}{Problem \arabic{#1} continued on next page\ldots}\nobreak{}
    \stepcounter{#1}
    \nobreak\extramarks{Problem \arabic{#1}}{}\nobreak{}
}

\setcounter{secnumdepth}{0}
\newcounter{partCounter}
\newcounter{homeworkProblemCounter}
\setcounter{homeworkProblemCounter}{1}
\nobreak\extramarks{Problem \arabic{homeworkProblemCounter}}{}\nobreak{}

%
% Homework Problem Environment
%
% This environment takes an optional argument. When given, it will adjust the
% problem counter. This is useful for when the problems given for your
% assignment aren't sequential. See the last 3 problems of this template for an
% example.
%
\newenvironment{homeworkProblem}[1][-1]{
    \ifnum#1>0
        \setcounter{homeworkProblemCounter}{#1}
    \fi
    \section{Problem \arabic{homeworkProblemCounter}}
    \setcounter{partCounter}{1}
    \enterProblemHeader{homeworkProblemCounter}
}{
    \exitProblemHeader{homeworkProblemCounter}
}

%
% Homework Details
%   - Title
%   - Due date
%   - Class
%   - Section/Time
%   - Instructor
%   - Author
%

\newcommand{\hmwkTitle}{CSE 547 Homework 2}
\newcommand{\hmwkDueDate}{April 26, 2017}
\newcommand{\hmwkClass}{CSE 547}
\newcommand{\hmwkAuthorName}{Brian de Silva}

%
% Title Page
%

\title{
    \vspace{2in}
    \textmd{\textbf{\hmwkClass:\ \hmwkTitle}}\\
    \normalsize\vspace{0.1in}\small{Due\ on\ \hmwkDueDate\ }\\
    \vspace{3in}
}

\author{\textbf{\hmwkAuthorName}}
\date{}

\renewcommand{\part}[1]{\textbf{\large Part \Alph{partCounter}}\stepcounter{partCounter}\\}

%
% Various Helper Commands
%

% Useful for algorithms
\newcommand{\alg}[1]{\textsc{\bfseries \footnotesize #1}}

% For derivatives
\newcommand{\deriv}[1]{\frac{\mathrm{d}}{\mathrm{d}x} (#1)}

% For partial derivatives
\newcommand{\pd}[2]{\frac{\partial}{\partial #1} (#2)}

\newcommand{\pdd}[2]{\frac{\partial #1}{\partial #2}}

% Integral dx
\newcommand{\dx}{\mathrm{d}x}

% Alias for the Solution section header
\newcommand{\solution}{\textbf{\vskip 0.2cm \large Solution:\\}}

% Useful commands
\newcommand{\bbm}{\begin{bmatrix}}
\newcommand{\ebm}{\end{bmatrix}}
\newcommand{\R}{\mathbb{R}}
\newcommand{\dtdx}{\frac{\Delta t}{\Delta x}}
\newcommand{\half}{\frac12}
\newcommand{\norm}[1]{\left\|#1\right\|}
% \newcommand{\Pr}{\mathbb{P}}


\begin{document}

\maketitle

\pagebreak

\section*{Collaborators}
I collaborated with Weston Barger and Emily Dinan on ***

% Problem 1
\begin{homeworkProblem}
	{\bf Gaussian Random Projections and Inner Products [10 Points]}
	\\
	\\
	In this problem, you will show that inner products are approximately preserved using random projections. Let $\phi(x) = \dfrac{1}{\sqrt{m}}Ax$ represent our random projection of $x \in \mathbb{R}^d$, with $A$ an $m \times d$ projection matrix with each entry sampled i.i.d from $N(0,1)$. (Note that each row of $A$ is a random projection vector, $v^{(i)}$.)  

	The \emph{norm preservation theorem} states that for all $x \in \mathbb{R}^d$, the norm of the random projection $\phi(x)$ approximately maintains the norm of the original $x$ with high probability:
	\begin{equation}
	\label{eq:npt}
	\Pr\left((1 - \epsilon) \norm{x}^2 \leq \norm{\phi(x)}^2 \leq (1 +
	\epsilon)\norm{x}^2 \right) \geq 1 - 2e^{-(\epsilon^2 -\epsilon^3)m/4},
	\end{equation}
	where $\epsilon \in (0, 1/2)$. 

	Using the norm preservation theorem, prove that for any $u,v \in \mathbb{R}^d$ s.t. $\norm{u} \leq 1$ and $\norm{v} \leq 1$,
	\begin{equation}
	\Pr(|u\cdot v - \phi(u)\cdot\phi(v)| \geq \epsilon) \leq 4e^{-(\epsilon^2 -\epsilon^3)m/4}.
	\end{equation}
	Note that $u\cdot v$ is the original dot product, and $\phi(u)\cdot \phi(v)$ is the dot product for the random projections. This statement puts a probabilistic bound on the distance between the two dot products. (\emph{Hint: Think about using Theorem (\ref{eq:npt}) with $x = u+v$ and $x = u-v$}).

	\solution

	First notice that
	\begin{align*}
		&(1-\epsilon)\|x\|^2\leq \norm{\phi(x)}^2\leq (1+\epsilon)\norm{x}^2\\
		\iff& \norm{x}^2-\epsilon\norm{x}\leq \norm{\phi(x)}^2\leq \norm{x}^2+\epsilon\norm{x}^2\\
		\iff & -\epsilon\norm{x}^2\leq \norm{\phi(x)}^2-\norm{x}^2\leq \epsilon\norm{x}^2\\
		\iff & \left|\norm{\phi(x)}^2-\norm{x}^2 \right|\leq \epsilon\norm{x}^2.
	\end{align*}
	Multiplying the above by $-1$ gives a similar statement:
	\[
		-\epsilon\norm{x}^2\leq \norm{x}^2- \norm{\phi(x)}^2\leq \epsilon\norm{x}^2.
	\]
	Observe also that $\phi$ is a linear function, so for any $x,y\in\R^d$, $\phi(x+y)=\phi(x)+\phi(y)$. We will need this fact later.
	Following the hint and applying Theorem (\ref{eq:npt}) to $x=u+v$ and $x=u-v$, we obtain
	\[
		\Pr\left((1-\epsilon)\|u+v\|^2\leq \norm{\phi(u+v)}^2\leq (1+\epsilon)\norm{u+v}^2\right) \geq 1-2e^{-(\epsilon^2-\epsilon^3)m/4}
	\]
	and
	\[
		\Pr\left((1-\epsilon)\|u-v\|^2\leq \norm{\phi(u-v)}^2\leq (1+\epsilon)\norm{u-v}^2\right) \geq 1-2e^{-(\epsilon^2-\epsilon^3)m/4}.
	\]
	Manipulating the inequalities as above, these two statements become
	\[
		\Pr\left(\left|\norm{\phi(u+v)}^2-\norm{u+v}^2 \right|\leq \epsilon\norm{u+v}^2 \right)\geq 1-2e^{-(\epsilon^2-\epsilon^3)m/4}
	\]
	and
	\[
		\Pr\left(\left|\norm{\phi(u-v)}^2-\norm{u-v}^2 \right|\leq \epsilon\norm{u-v}^2 \right)\geq 1-2e^{-(\epsilon^2-\epsilon^3)m/4}.
	\]
	These expressions are equivalent to the following
	\[
		\Pr\left(\left|\norm{\phi(u+v)}^2-\norm{u+v}^2 \right|\geq \epsilon\norm{u+v}^2 \right)\leq 1 - \left(1-2e^{-(\epsilon^2-\epsilon^3)m/4}\right) = 2e^{-(\epsilon^2-\epsilon^3)m/4}
	\]
	and
	\[
		\Pr\left(\left|\norm{\phi(u-v)}^2-\norm{u-v}^2 \right|\geq \epsilon\norm{u-v}^2 \right)\leq 1 - \left(1-2e^{-(\epsilon^2-\epsilon^3)m/4}\right) = 2e^{-(\epsilon^2-\epsilon^3)m/4}.
	\]

	Now, recall that for any two events $A$ and $B$, $\Pr(A\wedge B)\leq \Pr(A)+\Pr(B)$. Letting $A$ be the event that $$A:~~\left|\norm{\phi(u+v)}^2-\norm{u+v}^2 \right|\geq \epsilon\norm{u+v}^2$$ and $B$ be the event $$B:~~\left|\norm{\phi(u-v)}^2-\norm{u-v}^2 \right|\geq \epsilon\norm{u-v}^2,$$ it follows that the probability of both occuring is at most $\Pr(A)+\Pr(B)\leq4e^{-(\epsilon^2-\epsilon^3)m/4}$. We saw previously that 
	\begin{align*}
		&\left|\norm{\phi(u+v)}^2-\norm{u+v}^2 \right|\geq \epsilon\norm{u+v}^2\\
		\iff & -\epsilon\norm{u+v}^2\leq \norm{u+v}^2-\norm{\phi(u)+\phi(v)}^2\leq \epsilon\norm{u+v}^2
	\end{align*}
	and 
	\begin{align*}
		&\left|\norm{\phi(u-v)}^2-\norm{u-v}^2 \right|\geq \epsilon\norm{u-v}^2\\
		\iff & -\epsilon\norm{u-v}^2\leq \norm{\phi(u)-\phi(v)}^2 - \norm{u-v}^2\leq \epsilon\norm{u+v}^2.
	\end{align*}
	If both inequalities hold simultaneously, we may add them together to obtain
	\[
		-\epsilon\left(\norm{u+v}^2+\norm{u-v}^2\right) \leq \norm{\phi(u)-\phi(v)}^2-\norm{\phi(u)+\phi(v)}^2 + \norm{u+v}^2-\norm{u-v}^2\leq\epsilon\left(\norm{u+v}^2+\norm{u-v}^2\right).
	\]
	Using the Parallelogram law and Polarization identity, we can simplify this to
	\begin{align*}
		&-2\epsilon\left(\norm{u}^2+\norm{v}^2\right)\leq 4\left(u\cdot v-\phi(u)\cdot\phi(v) \right) \leq 2\epsilon\left(\norm{u}^2+\norm{v}^2\right)\\
		\iff& \epsilon\left(\norm{u}^2+\norm{v}^2\right)\leq 2\left(u\cdot v-\phi(u)\cdot\phi(v) \right) \leq \epsilon\left(\norm{u}^2+\norm{v}^2\right) \\
		\implies & 2\left|u\cdot v-\phi(u)\cdot\phi(v) \right|\leq \epsilon \left(\norm{u}^2+\norm{v}^2\right) \leq \epsilon(1+1) = 2\epsilon\\
		\implies& \left|u\cdot v-\phi(u)\cdot\phi(v) \right| \leq \epsilon.
	\end{align*}
	Putting everything together, we have our result
	\begin{align*}
		\Pr\left(\left|u\cdot v-\phi(u)\cdot\phi(v) \right| \leq \epsilon \right) \leq 4e^{-(\epsilon^2-\epsilon^3)m/4}.
	\end{align*}
\end{homeworkProblem}

\end{document}